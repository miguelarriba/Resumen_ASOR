%%%%%%%%%%%%%%%%%%%%%%%%%%%%%%%%%%%%%%%%%%%%
%        TeFloN 1.0
%
% Plantilla para TFGs básica:
%
% title.tex dentro de la carpeta data
% Este documento incluye todo lo necesario de la plantilla
% Se basa en TeXiS de una forma más básica para usar en TFGs
% 
% Los datos modificables están marcados
%
%%%%%%%%%%%%%%%%%%%%%%%%%%%%%%%%%%%%%%%%%%%%




\thispagestyle{empty}
\vspace*{9mm}
\begin{center}
    \rule[0.5ex]{\linewidth}{2pt}\vspace*{-\baselineskip}\vspace*{4.2pt}\\
    \rule[0.5ex]{\linewidth}{1pt}\\
    [3.1mm]
    {\textbf{\LARGE{Ampliación de Sistemas Operativos Y Redes}} }\\[3mm] %Titulo del TFG
    
    \rule[0.5ex]{\linewidth}{1pt}\vspace*{-\baselineskip}\vspace{4.2pt}
    \rule[0.5ex]{\linewidth}{2pt}\\
    \vspace{13mm}
    {\large\textsc{}}\\ %Nombres de los autores
    \vspace{11mm}
    \includegraphics[scale=0.3]{img/logo_UCM.jpg}\\ %Logo de la UCM
    \vspace{6mm}
    {\large Grado en Ingeniería Informática\\    %Departamento del profesor que es el tutor
    \textsc{Facultad de informática}}\\ %Facultad para la que se hace
    \vspace{19mm}
    \begin{minipage}{10cm}
    \begin{center}
      \textbf{Apuntes Teóricos}\\
      \vspace{2mm}
    \end{center}
    \vspace{4mm}
    
    %Cuadro de presentación del TFG
    \end{minipage}\\
    \vspace{12mm}
\end{center}